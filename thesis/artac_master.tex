\documentclass[]{article}

%imports
\usepackage{amsmath}
\usepackage{amsthm}
\usepackage[style=alphabetic]{biblatex}

%references
\addbibresource{ref.bib}
%\bibliography{ref.bib}

%newtheorems
\theoremstyle{definition}
\newtheorem{definition}{Definition}[section]

%commands
\newcommand{\mo}{\ensuremath{\text{mod }}}



%opening
\title{Reduction Techniques and other Combinatorial Tools in tau-tilting Theory}
\author{Håvard Utne Terland}

\begin{document}

\maketitle

\begin{abstract}

\end{abstract}

\section{Introduction}


\subsection{Setting}
An algebra $A$ will always refer to a finite dimensional algebra over a field $k$, which need not be algebraically closed. All algebras encountered in this thesis will be, unless otherwise stated, on the form $k\Gamma/I$ for some finite quiver $\Gamma$ and admissible ideal $I$. The number $n$ is, unless otherwise stated, reserved for the number of indecomposable projective summands of $A$. By $\mo A$.


\section{Tau-tilting theory}
$\tau$-tilting theory was introduced in \cite{tau} as a possible generalization of tilting theory, which has had tremendous impact on the field of representation theory of finite dimensional algebras as a whole. Although first introduced in \cite{tau}, some of the main results of $\tau$-tilting theory stem from (cite smalø). We here recall some important definitions and results from $\tau$-tilting theory.


\begin{definition}
	s
\end{definition}


\section{Tau-exceptional sequences}
Tau-exceptional sequences were introduced in \cite{buantau2020} as a generalization of exceptional sequences to any finite dimensional algebra.





\printbibliography

\end{document}
