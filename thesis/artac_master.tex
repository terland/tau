\documentclass[]{article}

%imports
\usepackage{amsmath}
\usepackage{amsthm}
\usepackage[style=alphabetic]{biblatex}

%references
\addbibresource{ref.bib}
%\bibliography{ref.bib}

%newtheorems
\theoremstyle{definition}
\newtheorem{definition}{Definition}[section]

\newtheorem{theorem}{Theorem}[section]
\newtheorem{proposition}{Proposition}[section]
\newtheorem{example}{Example}[section]

%commands
\newcommand{\mo}{\ensuremath{\text{mod }}}
\newcommand{\Hom}{\ensuremath{\text{Hom}}}
\newcommand{\tu}{\ensuremath{\tau}}
\newcommand{\Fac}{\ensuremath{\text{Fac}}}



%opening
\title{Reduction Techniques and other Combinatorial Tools in \tu-tilting Theory}
\author{Håvard Utne Terland}

\begin{document}

\maketitle

\begin{abstract}

\end{abstract}

\section{Introduction}


\subsection{Setting}
We closely follow the notation of \cite{assem_skowronski_simson_2006}. An algebra $A$ will always refer to a finite dimensional algebra over a field $k$, which need not be algebraically closed. All algebras encountered in this thesis will be, unless otherwise stated, on the form $k\Gamma/I$ for some finite quiver $\Gamma$ and admissible ideal $I$. The number $n$ is, unless otherwise stated, reserved for the number of indecomposable projective summands of $A$. By $\mo A$ we denote the category of finite dimensional left $A$-modules (in contrast to \cite{assem_skowronski_simson_2006}, who employ right modules). We denote by $D$ the duality on modules, and by $\text{Tr}$ the transpose. The Auslander-Reiten translate $D\text{Tr}$ is denoted $\tau$.

A module $M$ is called basic if no two nonzero summands are equal, that is $M \cong A \oplus X \oplus X$ implies $X = 0$. We denote by $|M|$ the number of indecomposable summands of $M$. 

\subsection{Approximations}
TODO: add note on approximations


\section{Tau-tilting theory}
$\tau$-tilting theory was introduced in \cite{tau} as a possible generalization of tilting theory, which has had tremendous impact on the field of representation theory of finite dimensional algebras as a whole. Although first introduced in \cite{tau}, some of the main results of $\tau$-tilting theory stem from (cite smalø). We here recall some important definitions and results from $\tau$-tilting theory. Also, we note that there are other approaches to defining \tu-rigid pairs which are fruitful in other contexts. To make these introductory sections both concise and self-contained, we will not dwell on these alternative approaches to the theory here.  When we need different but equivalent definitions in later sections, references to the literature will be made.

We start by discussing $\tau$-rigid modules, which may be considered the building blocks of $\tau$-tilting theory. 


\begin{definition}
	A module $M$ in $\mo A$ is called $\tau$-rigid if $\Hom(M,\tau M) = 0$.
\end{definition}

First note that the Auslander-Reiten formula gives that \tu-rigid objects must be exceptional objects. Note that $X,Y$ being $\tau$-rigid does not imply $X \oplus Y$ being $\tau$-rigid. In fact, studying how indecomposable $\tau$-rigid modules together form larger $\tau$-rigid modules is part of $\tau$-tilting theory.

\begin{definition}\cite[Definition 0.3]{tau}
	A pair $(M,P)$ where $M$ is basic $\tau$-rigid and $P$ basic projective such that $\Hom_A(P,M) = 0$ is called a  $\tau$-rigid pair.
\end{definition}

Along with this definition follows many conventions. For a  $\tau$-tilting module $T = (M,P)$, $P$ will be denote the projective part of $T$ and $M$ the module part. $T$ is called support \tu-tilting if $|M| + |P| = n$, and almost complete support \tu-tilting if $|M| + |P| = n - 1$. A support \tu-tilting pair with no non-zero projective part may also be denoted a \tu-tilting module. 

For $T_1,T_2$ both \tu-rigid pairs, we consider $T_1$ a summand of $T_2$ is both the module part and the projective part of $T_1$ is a summand of the module part and the projective part of $T_2$ respectively.


\subsection{Torsion classes induced by support \tu-modules}
Given a support \tu-tilting pair $(M,P)$, we may consider the modules generated by $M$, $\Fac M$. As in classical tilting theory, this is a torsion class, and we have the following very important result.

\begin{theorem}\cite[Theorem 2.7]{tau}\cite{auslandersmalo81}
	There is a bijection between the set of functorially finite torsion pairs, denoted $\text{f-tors} A$ and support \tu-tilting modules, denoted $\text{s}\tu\text{-tilt} A$, given by sending a support \tu-tilting pair $(M,P)$ to $(\Fac M,M^\perp)$.
\end{theorem}

We order the objects in $\text{f-tors} A$ by inclusion on the torsion classes. Thus $(X_1,Y_1) > (X_2,Y_2)$ if $X_2 \subseteq X_1$. This induces an ordering on $\text{s}\tu A$. We denote by $Q(\text{f-tors} A)$ the Hasse-quiver of the above defined ordering on $\text{f-tors} A$.

\subsection{Mutation}
An essential property of support \tu-tilting modules is that one can \textit{mutate} at each summand. More precisely, one has the following.

\begin{theorem}
	For a almost complete support $\tau$-tilting pair $(M,P)$, there exists exactly two distinct support \tu-tilting pairs $(M_0,P_0)$ and $(M_1,P_1)$. 
\end{theorem}

This implies that given a support \tu-tilting module over an algebra with $n$ projective modules, one can mutate to $n$ other support tilting modules. Inductively, one can then define a mutation graph where edges are induced by mutation. We postpone this slight

\subsection{g-vectors}
For an indecomposable module $M$, let $P_1 \to P_2 \to M \to 0$ be a minimal projective presentation of $M$, where $P_1 \cong \bigoplus_{i = 1}^n P(i)^{v_i}$ and $P_2 \cong \bigoplus_{i = 1}^n P(i)^{u_i}$. We may consider $(v_1,v_2,\dots,v_n)$ and $(u_1,u_2,\dots,u_n)$ to be vectors describing the building blocks of $P_1$ and $P_2$ respectively. Then we define $g^M$ to be $v - u$. 

\begin{definition}
	For a $\tau$-rigid pair $(M,P)$, its $g$-vector is defined as $g^M - g^P$. 
\end{definition}

Further, we follow \cite{schroll2020tautilting} and define $G$-matrices.

\begin{definition}\cite[Definition 2.4]{schroll2020tautilting}
	For a support \tu-tilting pair $(M,P) = (M_1 \oplus \dots \oplus M_i,P_1 \oplus \dots \oplus P_j)$, we define its G-matrix $G_{(M,P)}$ to be the $n \times n$ matrix $(g^{M_1},g^{M_2},\dots,g^{M_i},-g^{P_1},\dots,-g^{P_j})$, where we consider the $g$-vectors to be column vectors.
	
\end{definition} 

An important property of $g$-vectors is that they uniquely identify $\tau$-rigid pairs\cite[Theorem 5.5]{tau}. This immediately gives a proof that the set of indecomposable \tu-rigid objects must be countable, as there are only countably many integer vectors of dimension $n$. $G$-matrices of support \tu-tilting pairs are invertible. This follows immediately from \cite[Theorem 5.1]{tau}.



\section{Tau-exceptional sequences}
Tau-exceptional sequences were introduced in \cite{buantau2020} as a generalization of exceptional sequences to any finite dimensional algebra.





\printbibliography

\end{document}
